%
% File acl2014.tex
%
% Contact: koller@ling.uni-potsdam.de, yusuke@nii.ac.jp
%%
%% Based on the style files for ACL-2013, which were, in turn,
%% Based on the style files for ACL-2012, which were, in turn,
%% based on the style files for ACL-2011, which were, in turn, 
%% based on the style files for ACL-2010, which were, in turn, 
%% based on the style files for ACL-IJCNLP-2009, which were, in turn,
%% based on the style files for EACL-2009 and IJCNLP-2008...

%% Based on the style files for EACL 2006 by 
%%e.agirre@ehu.es or Sergi.Balari@uab.es
%% and that of ACL 08 by Joakim Nivre and Noah Smith

\documentclass[11pt]{article}
\usepackage{acl2014}
\usepackage{times}
\usepackage{url}
\usepackage{latexsym}
\usepackage{graphicx}
\usepackage{amsmath}
\usepackage[]{algorithm2e}

\DeclareGraphicsExtensions{.pdf,.png,.jpg}

%\setlength\titlebox{5cm}

% You can expand the titlebox if you need extra space
% to show all the authors. Please do not make the titlebox
% smaller than 5cm (the original size); we will check this
% in the camera-ready version and ask you to change it back.


\title{High-Dim Project}

\author{name1 \\
  Columbia University\\
  {\small \tt uni1@columbia.edu} \\\And
  name2 \\
  Columbia University\\
  {\small \tt uni2@columbia.edu} \\\And
  name3\\
  Columbia University\\
  {\small \tt uni3@columbia.edu} \\}
\date{}

\begin{document}
\maketitle
%\begin{abstract}
%  This document contains the instructions for preparing a camera-ready
%  manuscript for the proceedings of ACL-2014. The document itself
%  conforms to its own specifications, and is therefore an example of
%  what your manuscript should look like. These instructions should be
%  used for both papers submitted for review and for final versions of
%  accepted papers.  Authors are asked to conform to all the directions
%  reported in this document.
%\end{abstract}

\section{Introduction}

\section{Background}


\subsection{Multiclass Classification with Group Lasso}

The task of multiclass classification involves the prediction of a class label
$l$ where the number of possible labels is $k>2$. More often than not, the 
original problem is transformed into $k$ binary classification problems, i.e.
$1$--vs.--all classification and positive prediction with the highest 
confidence is selected as the label. This approach has the disadvantage of
having to train $k$ different models.

An alternative formulation, direct multiclass classification, tackles this 
problem directly by solving the following $\operatorname{argmax}$ problem:
$$y_i = \operatorname{argmax}_c W_{:c}^T x_i $$
where $W \in \mathcal{R}^{p \times k}$ is a weight matrix, with $W_{ij}$
corresponding to the $i$-th feature of class $j$. In this paper, we refer
to features as elements in the instance data $x$. A feature in $x$ is
associated with $k$ weights in $W$, one for each class.


The decision function above suggests a max-margin style loss function. More
specifically, we use the squared hinge loss:

$$ l(W) = \sum_{i=1}^n \sum_{r\ne y_i}^k 
    \max\left(1 - (W_{:y_i}^Tx_i - W_{:r}^Tx_i), 0\right)^2 $$

The minimization of $l$ directly will lead to a minimizer $W^*$ that is
dense. Sparse solutions are often explicitly sought, with model compactness
leading to fast prediction at test time. In order to obtain a sparse $W^*$,
a regularization term $r(W)$ is often applied, yielding the objective 
function:

$$\min_W l(W) + r(W).$$

Many choices are available for the regularizer $r$. In (ref ???), they use
the group lasso, where each row in $W$ is a group. The associated regularizer 
then is $r(W) = \lambda\sum_j^p \|W_{j:}\|_2$ where $\lambda$ is a parameter
that adjusts the strength of the regularization.
This has the effect of producing a few rows of non-zero values in $W$;
since each row corresponds to an individual feature, the optimal sparse $W^*$
yields a fast-evaluating decision function, i.e. most features are 
ignored at test time.

To minimize this multiclass classification group lasso objective, ??? use 
coordiante descent, iteratively solving a sub-problem with respect to a 
single group. Figure ? shows a general outline of algorithm that involves
computing the partial gradient with respect to the 
current group $j$, the prox
operator of the L2 norm, and a final line search to identify an appropriate
step size for the current update.

\begin{algorithm}
 \For{$i \gets 1, \ldots, max\;iters $}{
     \For{$j \gets 1, \ldots, p$}{
        Compute gradient $l^\prime(W)_{j:}$\\
        Choose $\mathcal{L}_j$\\
        Compute \\
        $\;\;\;\;V_j = W_{j:} - \frac{1}{\mathcal{L}_j}l^\prime(W)_{j:}$\\
        $\;\;\;\;W_{j:}^* 
          = \operatorname{Prox}_{\frac{\lambda}{\mathcal{L}_j}\|\cdot\|_j}
         (V_j)$\\

         $\;\;\;\;\delta = W_{j:}^* - W_{j:}$\\
        Choose $\alpha$\\
        $W_{j:} \gets W_{j:} + \alpha \delta$\\

    }
 }
\end{algorithm}

Efficient computation of this objective is possible by storing current 
loss for each data point. Let $A$ be an $n\times k$ matrix where the 
$i,r$-th element corresponds to 
 $(1-(W_{:y_i}^Tx_i - W_{:r}x_i).$
 The gradient can then be calculated as 
 $l^\prime(W)_{j} = \frac{2}{n}\sum_{i=1}^n\sum_{r\ne y_i}
 \max(A_{ir}, 0)(x_{ij}e_{y_i} - x_{ij} e_r)$
where $e_r$ is a $k$ dimensional vector with zeroes everywhere except for a 1
at the $r$-th position. We only have to examine elements in $A$ for which 
the corresponding $x_{ij}$ is non-zero. When $x_{i}$ is sparse, more often
than not $x_{ij}$ is zero and can be ignored.



\subsection{Latent Group Lasso}

One limitation of group lasso is that it assumes that group assignments are
non-overlapping. In some domains, this can be too restrictive an assumption.
For example, in document classification, individual words are used as features.
If we were to construct groupings of these features, we might run into a case
where one word could reasonably be added to several groups. The overlapping or
latent group lasso was introduced to handled such cases.

??? develop a theoretical justification for the latent group lasso, as well
as its equivalence to a regular group lasso in a higher dimensional space.
Let $\mathcal{G}$ be the set of (possibly overlapping) groups,
where $g \in \mathcal{G}$ is a set of indices of covariates associated with 
that group. 
Let our
data consist of vectors $x_i$ in $p$ dimensions, and let $w$ be the 
corresponding weight vector in $p$ dimensions that we would like to learn.
Finally, define $\operatorname{supp}(v)$ to be the support of $v$, i.e.
the indices of the non-zero elements in $v$.


For each group
$g \in \mathcal{G}$ we associate a latent vector $v^g \in \mathcal{R}^p$ where
$\operatorname{supp}(v^g) = g$, i.e. the nonzero elements in the $v^g$ correspond
to the indices in the group $g$. 
The original weight vector $w$ can be
interpreted as a sum of the latent vectors, or
$w = \sum_{g \in \mathcal{G}} v^g$. 
??? arrive at the following minimization
problem 
$$ \min_{w, v^g} l(w) + \lambda \sum_{g \in \mathcal{G}} d_g \|v^g \|_2 $$
$$\mathrm{s.t.}\;\;\; w = \sum_g v^g$$ 



??? show that when the original problem is regression, 
$w^Tx = \left(\sum_g v^g\right)^Tx = \hat{v}^T \hat{x}$ where 
$\hat{v} = (v^{g})_{g \in \mathcal{G}}$ and
$\hat{x} = \bigoplus_{g \in \mathcal{G} } (x_i)_{i \in g}$, i.e. $\hat{x}$ is
the restrictions of each $g$ stacked on top of each other. $\hat{x}, \hat{v}$ 
have dimension $\sum_{g \in \mathcal{G}} |g|$ 
In this formulation, the optimal $\hat{v}^*$ can be found using regular 
non-overlapping group lasso. 



\section{Our Model}

Given n training vectors $x_i \in R_d$ and their class labels $y_i \in \{1, ..., m\}$, our goal is to compute $W$ such that it maximizes the accurarcy of our prediction and it is group-wise sparse. \\

In our model, we minimize the following objective function : \\

$$ \min_{W \in R^{d x m}} F(W) = $$
$$\frac{1}{n} \sum_{i=1}^{n} \sum_{r \neq y_i } \max(1 - ( W_{:y_i}^T \cdot x_i - W_{:r}^T \cdot x_i) , 0 )^2 $$
$$ + \lambda \sum_{g \in \mathcal{G}} \sum_{r=1}^{d} \| W_{g,r} \|_2$$ \\

The first term is the multiclass squared hinge loss function. We want the dot product of an instance and its feature vector to be as large as possible, and the dot product of this instance and the rest feature vectors to be as small as possible. And as long as their difference is greater then a margin ($1$ in this case), we won't penalize it. In the second term, $W_{g,r}$ means a block of weights in group $g$ and class $m$. The L2-norm regulization is computed and sum up for each block. The $\lambda > 0$ is a parameter controls the trade-off between the hinge loss and the L2-norm regulization.  \\



\section{Data}

\subsection{Newsgroup Data}

\subsubsection{Group Identification}

\subsection{Artificial Data}

For the datasets described above, we can't tell with 100 percent confidence that the datasets follow the assumptions of the group structures for the features. And even if they are indeed structured that way, we maybe wrong with the method of coming up with the groups. These issues make it difficult to access our model.\\ 

To get rid of all these problems and validate the effectiveness of our model, we created artificial data that followed the underlying assumptions of the model. First, we generate a sparse weight matrix W to represent the relationship between features and classes. The weight matrix W has an internal structure in which features are grouped together. And also, only a small number of groups have non-zero weights. This makes the matrix sparse.\\ 

\begin{figure}[ht]
\begin{center}
	\includegraphics[width=5cm, natwidth=200, natheight=206]{m_img}
	\caption{Group-wise sparse weight matrix generated: 5 classes, 25 features in 5 groups}
\end{center}
\end{figure}

Then we generate random vectors, each of which has a length of the number of all features, and calculate dot product with the weight matrix W to get the class assignments for these random vectors. The random vetors X and the class assignments Y make up the training data set. \\

Our goal is to infer this weight matrix W from X and Y using our model. By generating the data set using this method, we can test the effectiveness of our model on a noiseless dataset with right underlying assumptions.\\ 


\section{Results}

\subsection{Newsgroup Data}

\subsection{Artificial Data}

Shape Recovery. One of the main indicator of the effectiveness of our model is to see whether the calculated weight matrix is sparse group-wise. Our experiments show so. The following figures shows in a typical trial, the generated target weight matrix and the recovered weight matrix by our model. By comparing them side by side, we can tell that their sparsity patterns are similar. 

\begin{figure}[ht]
\begin{center}
	\includegraphics[width=5cm, natwidth=200, natheight=307]{m1_img}
	\caption{Target group-wise sparse weight matrix generated: 5 classes, 25 features in 5 groups}
\end{center}
\end{figure}

\begin{figure}[ht]
\begin{center}
	\includegraphics[width=8cm,natwidth=200, natheight=186]{m2_img}
	\caption{Weight matrix calculated from the training data. It's similar to the one generated.}
\end{center}
\end{figure} 

The different weights between the calculated matrix and the target matrix can be due to many factors. First, sample coverage is a major factor. In our simulation data, the sample size is small. Limited by the time complexity of the algorithm, it's difficult to complete the computation for a very large sample size in a reasonable time. Also, the sampling is random. There is no guaranteed that the inferred weights leads to the target weights. \\   

Accuracy. In our experiments, the generator algorithm was configured to produce 150 random vectors from the underlying model where it consists of 5 classes and 25 features in 5 groups. The accuracy achieved was about $60\%$ to $70\%$. \\

\section{Conclusion}
% include your own bib file like this:
%\bibliographystyle{acl}
%\bibliography{acl2014}

%\begin{thebibliography}{}
%
%\bibitem[\protect\citename{Aho and Ullman}1972]{Aho:72}
%Alfred~V. Aho and Jeffrey~D. Ullman.
%\newblock 1972.
%\newblock {\em The Theory of Parsing, Translation and Compiling}, volume~1.
%\newblock Prentice-{Hall}, Englewood Cliffs, NJ.
%
%\bibitem[\protect\citename{{American Psychological Association}}1983]{APA:83}
%{American Psychological Association}.
%\newblock 1983.
%\newblock {\em Publications Manual}.
%\newblock American Psychological Association, Washington, DC.
%
%\bibitem[\protect\citename{{Association for Computing Machinery}}1983]{ACM:83}
%{Association for Computing Machinery}.
%\newblock 1983.
%\newblock {\em Computing Reviews}, 24(11):503--512.
%
%\bibitem[\protect\citename{Chandra \bgroup et al.\egroup }1981]{Chandra:81}
%Ashok~K. Chandra, Dexter~C. Kozen, and Larry~J. Stockmeyer.
%\newblock 1981.
%\newblock Alternation.
%\newblock {\em Journal of the Association for Computing Machinery},
%  28(1):114--133.
%
%\bibitem[\protect\citename{Gusfield}1997]{Gusfield:97}
%Dan Gusfield.
%\newblock 1997.
%\newblock {\em Algorithms on Strings, Trees and Sequences}.
%\newblock Cambridge University Press, Cambridge, UK.
%
%\end{thebibliography}

\end{document}
