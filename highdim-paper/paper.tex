%
% File acl2014.tex
%
% Contact: koller@ling.uni-potsdam.de, yusuke@nii.ac.jp
%%
%% Based on the style files for ACL-2013, which were, in turn,
%% Based on the style files for ACL-2012, which were, in turn,
%% based on the style files for ACL-2011, which were, in turn, 
%% based on the style files for ACL-2010, which were, in turn, 
%% based on the style files for ACL-IJCNLP-2009, which were, in turn,
%% based on the style files for EACL-2009 and IJCNLP-2008...

%% Based on the style files for EACL 2006 by 
%%e.agirre@ehu.es or Sergi.Balari@uab.es
%% and that of ACL 08 by Joakim Nivre and Noah Smith

\documentclass[11pt]{article}
\usepackage{acl2014}
\usepackage{times}
\usepackage{url}
\usepackage{latexsym}

%\setlength\titlebox{5cm}

% You can expand the titlebox if you need extra space
% to show all the authors. Please do not make the titlebox
% smaller than 5cm (the original size); we will check this
% in the camera-ready version and ask you to change it back.


\title{High-Dim Project}

\author{name1 \\
  Columbia University\\
  {\small \tt uni1@columbia.edu} \\\And
  name2 \\
  Columbia University\\
  {\small \tt uni2@columbia.edu} \\\And
  name3\\
  Columbia University\\
  {\small \tt uni3@columbia.edu} \\}
\date{}

\begin{document}
\maketitle
%\begin{abstract}
%  This document contains the instructions for preparing a camera-ready
%  manuscript for the proceedings of ACL-2014. The document itself
%  conforms to its own specifications, and is therefore an example of
%  what your manuscript should look like. These instructions should be
%  used for both papers submitted for review and for final versions of
%  accepted papers.  Authors are asked to conform to all the directions
%  reported in this document.
%\end{abstract}

\section{Introduction}


\section{Background}

\subsection{Latent Group Lasso}

\subsection{Multiclass Classification with Group Lasso}


\section{Our Model}


\section{Data}

\subsection{Newsgroup Data}

\subsubsection{Group Identification}

\subsection{artificial data}

\section{Results}

\section{Conclusion}
% include your own bib file like this:
%\bibliographystyle{acl}
%\bibliography{acl2014}

%\begin{thebibliography}{}
%
%\bibitem[\protect\citename{Aho and Ullman}1972]{Aho:72}
%Alfred~V. Aho and Jeffrey~D. Ullman.
%\newblock 1972.
%\newblock {\em The Theory of Parsing, Translation and Compiling}, volume~1.
%\newblock Prentice-{Hall}, Englewood Cliffs, NJ.
%
%\bibitem[\protect\citename{{American Psychological Association}}1983]{APA:83}
%{American Psychological Association}.
%\newblock 1983.
%\newblock {\em Publications Manual}.
%\newblock American Psychological Association, Washington, DC.
%
%\bibitem[\protect\citename{{Association for Computing Machinery}}1983]{ACM:83}
%{Association for Computing Machinery}.
%\newblock 1983.
%\newblock {\em Computing Reviews}, 24(11):503--512.
%
%\bibitem[\protect\citename{Chandra \bgroup et al.\egroup }1981]{Chandra:81}
%Ashok~K. Chandra, Dexter~C. Kozen, and Larry~J. Stockmeyer.
%\newblock 1981.
%\newblock Alternation.
%\newblock {\em Journal of the Association for Computing Machinery},
%  28(1):114--133.
%
%\bibitem[\protect\citename{Gusfield}1997]{Gusfield:97}
%Dan Gusfield.
%\newblock 1997.
%\newblock {\em Algorithms on Strings, Trees and Sequences}.
%\newblock Cambridge University Press, Cambridge, UK.
%
%\end{thebibliography}

\end{document}
